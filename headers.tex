%%%%%%%%%%%%%%% INDEX %%%%%%%%%%%%%%%

\usepackage{imakeidx}

%%%%% Code magique pour faire marcher dotted_mode.ist
%% D'après https://tex.stackexchange.com/questions/594568/how-to-integrate-a-tabular-environment-in-the-index
\setlength{\columnseprule}{0.4pt}
\newsavebox\ltmcbox
\newlength\mysavecolroom
%% Utilisé dans les fichiers .ist afin d'ignorer automatiquement 
%% les instances de \\ dans les index qui emploient longtable, ce qui casserait la table.
\newcommand{\ignoreNL}[1]{} 
%% Macro à employer pour effectuer des sauts de lignes dans les noms indexés si ce nom est indexé avec longtable (p.ex. dotted_mode.ist)
\newcommand{\idxnewline}{\newline\null\hspace{4mm}}
%%%%% Fin du code magique

%% Définition des index 
\indexsetup{level=\section*,toclevel=section,noclearpage,othercode=\footnotesize\thispagestyle{empty}}
\makeindex[name=A,title=Antiphonæ, columns=2,columnseprule, options=-s dotted_mode.ist]
\makeindex[name=H,title=Hymni, columns=2,columnseprule, options=-s dotted_mode.ist]
\makeindex[name=R,title=Responsoria, columns=2,columnseprule, options=-s dotted_mode.ist]
\makeindex[name=F,title=Festa, columns=2,columnseprule, options=-s dotted.ist]

%%%%%%%%%%%%%%% GÉOMÉTRIE %%%%%%%%%%%%%%%

%KDP 6inx9in sans fonds perdus
\usepackage[paperwidth=6in, paperheight=9in]{geometry}

\geometry{
inner=25mm,
outer=14mm,
top=14mm,
bottom=14mm,
headsep=3mm,
}

%%%%%%%%%%%%%%% PACKAGES STANDARD %%%%%%%%%%%%%%%

\usepackage{fontspec}
\usepackage[nolocalmarks]{polyglossia}
%% utiliser la typographie francaise est une bonne idée même pour le latin (l'option latin existe mais marche moins bien)
\setdefaultlanguage[variant=french, frenchitemlabels=false]{french}
\usepackage[table]{xcolor}
\usepackage{fancyhdr}
\usepackage{titlesec}
\usepackage{setspace}
\usepackage{expl3}
\usepackage{hyperref}
\usepackage{refcount}
\usepackage{needspace}
\usepackage{etoolbox}
\usepackage{enumitem}
\usepackage{lettrine}
\usepackage{longtable}
\usepackage{luacode}
\usepackage{paracol}
\usepackage{alltt}

%%%%%%%%%%%%%%% POLICES DE CARACTÈRES %%%%%%%%%%%%%%%
\setmainfont[Ligatures=TeX]{Plantin MT Pro} % changer pour changer la police
\setstretch{1.05} % changer pour augmenter ou diminuer l'interligne

%%%%%%%%%%%%%%% CONFIGURATION GREGORIO %%%%%%%%%%%%%%%

\usepackage[forcecompile]{gregoriotex}

%% Variable globale contenant le nom de fichier de la pièce en train d'être imprimée, qui est aussi son label pour référence interne
\newcommand{\labelbuffer}{}
%% Variable globale contenant le nom de l'index (A, H, R) pertinent pour la pièce en train d'être imprimée, s'il y en a un
\newcommand{\indexnamebuffer}{}
%% Variable globale contenant le nom de la pièce (tel qu'indiqué dans l'en-tête name:) en train d'être imprimée
\newcommand{\namebuffer}{}
%% Variable globale contenant le mode (+ differentia si antienne) de la pièce en train d'être imprimée, pour l'imprimer en annotation
\newcommand{\annotationmode}{}
%% Variable globale contenant le mode (+ parfois differentia si un mode a plusieurs pointages) pour déterminer le prochain psaume pointé.
\newcommand{\nextpsalmmode}{}
%% Variable globale contenant le mode de la pièce en train d'être imprimée, pour l'ajouter à son entrée d'index.
\newcommand{\indexedmode}{}

%%%% Code magique pour définir \grenewcommand, qui fait la même chose que \renewcommand mais avec un effet global
\makeatletter
\def\gnewcommand{\g@star@or@long\new@command}
\def\grenewcommand{\g@star@or@long\renew@command}
\def\g@star@or@long#1{% 
  \@ifstar{\let\l@ngrel@x\global#1}{\def\l@ngrel@x{\long\global}#1}}
\makeatother
%%%% fin du code magique

%% Cette section prend un header office-part (Antiphona, etc.) détermine son index
\newcommand{\abbrev}[3]{%
  \ifstrequal{#1}{#2}{%
	\grenewcommand{\indexnamebuffer}{#3}%
  }%
  {}%
}
\newcommand{\officepartannotation}[1]{%
  \grenewcommand{\indexnamebuffer}{}%
  \abbrev{#1}{Antiphona}{A}%
  \abbrev{#1}{Hymnus}{H}%
  \abbrev{#1}{Responsorium breve}{R}%
  \abbrev{#1}{Responsorium}{R}%
}

%% Cette section prend un header mode (1a...), annote la pièce avec ce mode, 
%% et en prend note dans \nextpsalmmode pour aller chercher le bon psaume après.
%% étant la dernière section exécutée immédiatement avant l'impression de la pièce (car dernier header dans l'ordre),
%% elle porte également l'indexation avec le nom, le mode et l'index identifiés.
\newcommand{\changemode}[5]{%
  \ifstrequal{#1}{#2}{%
	\grenewcommand{\nextpsalmmode}{#3}%
	\grenewcommand{\annotationmode}{#4}%
	\grenewcommand{\indexedmode}{#5}%
  }%
  {}%
}
\newcommand{\modeannotation}[1]{
  % 1e colonne: champ mode du GABC
  % 2e colonne: mode de pointage du psaume
  % 3e colonne: mode imprimé au-dessus de l'initiale
  % 4e colonne: mode imprimé dans l'entrée d'index
  \grenewcommand{\annotationmode}{#1}
  \grenewcommand{\nextpsalmmode}{#1}
  \changemode{#1}{1a}{1}{I a}{1}
  \changemode{#1}{1b}{1}{I b}{1}
  \changemode{#1}{1c}{1}{I c}{1}
  \changemode{#1}{1}{1}{I}{1}
  \changemode{#1}{2}{2}{II}{2}
  \changemode{#1}{3a}{3}{III a}{3}
  \changemode{#1}{3b}{3}{III b}{3}
  \changemode{#1}{3}{3}{III}{3}
  \changemode{#1}{4a}{4}{IV a}{4}
  \changemode{#1}{4b}{4}{IV b}{4}
  \changemode{#1}{4}{4}{IV}{4}
  \changemode{#1}{5}{5}{V}{5}
  \changemode{#1}{6}{6}{VI}{6}
  \changemode{#1}{7a}{7}{VII a}{7}
  \changemode{#1}{7b}{7}{VII b}{7}
  \changemode{#1}{7}{7}{VII}{7}
  \changemode{#1}{8a}{8}{VIII a}{8}
  \changemode{#1}{8b}{8}{VIII b}{8}
  \changemode{#1}{8}{8}{VIII}{8}
  \changemode{#1}{Per.}{per}{Per.}{P}
  \greannotation{\annotationmode}
  %% indexation
  \ifstrequal{H}\expandafter{\indexnamebuffer}{%
    \index[\indexnamebuffer]{{\namebuffer}@\indexedmode & \namebuffer}%
  }{}%
  \ifstrequal{A}\expandafter{\indexnamebuffer}{%
    \index[\indexnamebuffer]{{\namebuffer}@\indexedmode & \namebuffer}%
  }{}%
  \ifstrequal{R}\expandafter{\indexnamebuffer}{%
    \index[\indexnamebuffer]{{\namebuffer}@\indexedmode & \namebuffer}%
  }{}%
}

%% Cette section prend un header name (le nom de la pièce),
%% assigne son label (le nom de fichier) à la pièce, et un nom au label (le nom de la pièce),
%% et mémorise le nom dans \namebuffer pour indexation future une fois qu'on a le mode.
\begin{luacode*}
function delete_newline ( s )
   s = string.gsub ( s, 'newline', '')
   s = string.gsub ( s, 'hspace', '')
   s = string.gsub ( s, '4mm', '')
   s = string.gsub ( s, 'protect', '')
   s = string.gsub ( s, 'hbox', '')
   s = string.gsub ( s, [[\]], '')
   s = string.gsub ( s, "{}", '')
   tex.sprint ( s )
end
\end{luacode*}
\makeatletter
\newcommand{\namecapture}[1]{%
  %% this prevents page breaks between the phantom section and its label, and the actual score.
  \needspace{4\baselineskip}%
  \protected@edef\@currentlabelname{\directlua{delete_newline(\luastring{#1})}}%
  \phantomsection%
  \label{\labelbuffer}%
  \grenewcommand{\namebuffer}{#1}
}
\makeatother

\gresetheadercapture{office-part}{officepartannotation}{}
\gresetheadercapture{mode}{modeannotation}{string}
\gresetheadercapture{name}{namecapture}{}

\newcommand{\cantusparvus}[1]{
  \gresetinitiallines{0}
  \gregorioscore{gabc/#1}
  \gresetinitiallines{1}
}

\newcommand{\cantus}[3][]{
  %% #1 (optionnel) : l'annotation à faire figurer à la place du mode
  %% #2 : le nom de fichier (sans .gabc) qui sera la référence de la pièce
  %% #3 : l'annotation à faire figurer au-dessus du mode
  \renewcommand{\labelbuffer}{#2}
  \greannotation{#3}
  \gregorioscore{gabc/#2}
}


%%%%%%%%%%%%%%% COLUMN MANAGEMENT %%%%%%%%%%%%%%%

%% Commande pour s'autoriser à avoir de l'espace horizontal entre
%% les mots au-delà des standards normalement acceptables
\sloppy

%% Commande pour imprimer une liste itémisée (psaumes, cantiques)
\newcommand{\twocolitemized}[2][]{
	%% #1 (optionnel) : titre à imprimer en en-tête
	%% #2 : fichier itémisé à imprimer sur deux colonnes
	\begin{multicols}{2}%
	\begin{center}
	#1
	\end{center}
	\begin{itemize}[
		label=\null, 
		leftmargin=0pt, 
		itemindent=5pt, 
		labelsep=0pt, 
		labelwidth=0pt, 
		rightmargin=0pt, 
		parsep=0pt, 
		itemsep=0pt plus 1pt]
	\input{#2}%
	\end{itemize}%
	\end{multicols}%
}

\newcommand{\psalmus}[2][\nextpsalmmode]{
	%% #1 (optionnel): ton du psaume si différent de \nextpsalmmode
	%% #2: numéro du psaume
	\phantomsection
	\label{Psalmus#2_#1}
	%\index[P]{#2 (tonus #1)} %% seulement si on indexe les psaumes
	\twocolitemized[\textbf{Psalmus #2}]{psalmi/#2_#1}
}

\newcommand{\canticum}[4][\nextpsalmmode]{
	%% #1 (optionnel): ton du cantique si différent de \nextpsalmmode
	%% #2: nom de fichier du cantique
	%% #3: titre du cantique tel qu'imprimé
	%% #4: référence biblique
	\phantomsection
	\label{Canticum#2_#1}
	%\index[C]{#2 (tonus #1)} %% seulement si on indexe les cantiques
	\twocolitemized[\textbf{#3}\\\rubric{#4}]{psalmi/#2_#1}
}

%%%%%%%%%%%%%%% EN-TÊTES DE PAGES %%%%%%%%%%%%%%%

\pagestyle{fancy}
\fancyhead{}
\fancyfoot{}
\renewcommand{\headrulewidth}{1pt}
\setlength{\headheight}{20pt}
\newcommand{\themark}{}
\fancyhead[RO, LE]{\thepage}
\fancyhead[CO, CE]{\themark}

% cette commande est appelée pour définir l'en-tête de page (p.ex. en appelant \festum)
% la surcharge de \hyphenpenalty est nécessaire sur certaines anciennes versions de gregorio
% voir https://tex.stackexchange.com/questions/581013/lualatex-hyphenation-issue-in-fancyhdr-with-gregoriotex-and-multicols-latin-te
\newcommand{\setheader}[1]{
	\grenewcommand{\themark}{\hyphenpenalty=50{\sc#1}}
	\grenewcommand{\themark}{{\sc#1}}
}

%%%%%%%%%%%%%%% STYLES %%%%%%%%%%%%%%%

%% Pas d'indentation des paragraphes
\setlength{\parindent}{0mm}

%% Macro to print the name of a score in normal characters inside a \rubric
\newcommand{\normaltext}[1]{{\normalfont\normalcolor #1}}
\newcommand{\scorename}[1]{\normaltext{\nameref{M-#1}}}

%% Macros to print preparatory and accented syllables in psalms
\newcommand{\sylac}[1]{\textbf{#1}}
\newcommand{\sylprep}[1]{\textit{#1}}

\newcommand{\sylchange}[1]{\textbf{#1}}
\newcommand{\sylpass}[1]{\textit{#1}}

%%%%%%%%%%%%%%% CARACTÈRES SPÉCIAUX %%%%%%%%%%%%%%%

%% V/, R/, A/, la croix, l'astérisque et l'obèle, pour usage dans GABC (<sp>V/</sp>, <sp>R/</sp>, <sp>A/</sp>, <sp>cross</sp>, <sp>*</sp>, <sp>+</sp>)
\gresetspecial{V/}{\textcolor{gregoriocolor}{\fontspec{Charis}℣.~}}
\gresetspecial{R/}{\textcolor{gregoriocolor}{\fontspec{Charis}℟.~}}
\gresetspecial{A/}{\textcolor{gregoriocolor}{\fontspec{Charis}\Abar.~}}
\gresetspecial{+}{{\fontspec{Plantin MT Pro}†~}}
\gresetspecial{*}{\gresixstar}
\gresetspecial{crux}{\textcolor{gregoriocolor}{\fontspec{FreeSerif}\symbol{"2720}}}
\gresetspecial{lcrux}{\textcolor{gregoriocolor}{+}}
%% Les mêmes, pour usage dans LaTeX (\vv \rr \aa \cc \psstar \pscross), pour usage dans le texte et les psaumes
\newcommand{\specialcharhsep}{3mm} % space after invoking R/ or V/ or A/ outside rubrics
\newcommand{\vv}{\textcolor{gregoriocolor}{\fontspec{Charis}℣.\nolinebreak[4]\hspace{\specialcharhsep}\nolinebreak[4]}}
\newcommand{\rr}{\textcolor{gregoriocolor}{\fontspec{Charis}℟.\nolinebreak[4]\hspace{\specialcharhsep}\nolinebreak[4]}}
\renewcommand{\aa}{\textcolor{gregoriocolor}{\fontspec{Charis}\Abar.\nolinebreak[4]\hspace{\specialcharhsep}\nolinebreak[4]}}
\newcommand{\cc}{\textcolor{gregoriocolor}{\fontspec{FreeSerif}\symbol{"2720}~}}
\newcommand{\psstar}{~\GreSpecial{*}}
\newcommand{\pscross}{~\GreSpecial{+}}
%% Les mêmes, pour usage dans les rubriques (pas d'espace avant, couleur identique au reste de la rubrique)
\newcommand{\vvrub}{{\fontspec{Charis}℣.~}}
\newcommand{\rrrub}{{\fontspec{Charis}℟.~}}
\newcommand{\aarub}{{\fontspec{Charis}\Abar.~}}

%%%%%%%%%%%%%%% INTERCALAIRES

%% ligne horizontale sur toute la largeur de la page
\newcommand{\lineamagna}{
	\begin{center}
	\noindent\rule{\textwidth}{0.4pt}
	\end{center}
}

%% ligne horizontale sous les titres
\newcommand{\lineaparva}{
	\begin{center}
	\noindent\rule{3cm}{0.4pt}
	\end{center}
}



%%%%%%%%%%%%%%% GRAPHIES SPÉCIFIQUES

%% Versicules
\newcommand{\versiculus}[2]{
	\vv #1 \\ \rr #2
}

%% Rubriques
\newcommand{\rubrica}[1]{\textcolor{gregoriocolor}{\emph{#1}}}

%% Rubriques employées comme titre d'une pièce ("per annum" centré)
\newcommand{\titulumrubrica}[1]{\begin{center}\textbf{#1}\end{center}}

%% Petit titre intermédiaire
\newcommand{\titulum}[1]{\begin{center}\textbf{#1}\end{center}}

%%%%%%%%%%%%%%% COMMON RUBRICS %%%%%%%%%%%%%%%

\newcommand{\dominexaudiversiculus}{%
	\versiculus{Dómine, exáudi oratiónem meam.}{Et clamor meus ad te véniat.}%
}

\newcommand{\dominusvobiscumversiculus}{%
	\versiculus{Dóminus vobíscum.}{Et cum spíritu tuo.}%
}

\newcommand{\festum}[5]{
	%% pré-titre, titre, sous-titre
	\begin{center}
	\ifstrequal{#5}{0}{
		\ifstrequal{#1}{}{}{\huge \textsc{#1}\par}
		{\Huge \textsc{#2}}\par
		{\huge \textsc{#3}}\par
	}{}
	\ifstrequal{#5}{1}{
		\ifstrequal{#1}{}{}{\Large \textsc{#1}\par}
		{\huge \textsc{#2}}\par
		{\Large \textsc{#3}}\par
	}{}
	\ifstrequal{#5}{2}{
		\ifstrequal{#1}{}{}{\large \textsc{#1}\par}
		{\LARGE \textsc{#2}}\par
		{\large \textsc{#3}}\par
	}{}
	\ifstrequal{#5}{3}{
		\ifstrequal{#1}{}{}{\normalsize \textsc{#1}\par}
		{\Large \textsc{#2}}\par
		{\normalsize \textsc{#3}}\par
	}{}
	\end{center}
	%% définition de l'en-tête de page
	\setheader{#4}
	%% si une date est indiquée, on indexe
	%\ifstrequal{#1}{}{}{\index[F]{#2}} %% pas beau: rajouter un argument pour le nom indexé de la fête, ou déduire ce nom en coupant S. devant
}

%%%%%%%%%%%%%%% SUBFILES %%%%%%%%%%%%%%%

\usepackage{xr}
\usepackage{subfiles}

%% When we start a new subfile (new chapter), 
%% we start on a new page (with blank filling on the previous page) and create a corresponding label.
\newcommand{\customsubfile}[1]{\newpage\label{#1}\thispagestyle{empty}\subfile{#1}}
