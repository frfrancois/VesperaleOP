% !TEX TS-program = lualatex
% !TEX encoding = UTF-8

\documentclass[VesperaleOP.tex]{subfiles}

\ifcsname preamble@file\endcsname
  \setcounter{page}{\getpagerefnumber{M-vop15_psalterium}}
\fi

\begin{document}

\festum{}{PROPRIUM DE TEMPORE}{}{}{0}
\lineaparva

%%%% DOM I. ADVENTUS
\festum{}{Dominica I Adventus}{I classis}{Dominica I Adventus}{2}
\titulumrubrica{Ad I Vesperas}
\cantus{A1F0VR}{\rrrub}
\cantus{AVH}{Hymnus}
\versiculus{Roráte, cæli, désuper, et nubes pluant justum.}{Aperiátur terra, et gérminet Salvatórem.}
\cantus{A1F0VAM}{Ad Magn.}

\titulumrubrica{Ad II Vesperas}
\cantus{A1F1VAM}{Ad Magn.}

%%%% INFRA HEBDOMADAM I ADVENTUS
\festum{}{Feria secunda}{}{Infra hebdomadam I Adventus}{3}
\cantus{A1F2VAM}{Ad Magn.}

\festum{}{Feria tertia}{}{Infra hebdomadam I Adventus}{3}
\cantus{A1F3VAM}{Ad Magn.}

\festum{}{Feria quarta}{}{Infra hebdomadam I Adventus}{3}
\cantus{A1F4VAM}{Ad Magn.}

\festum{}{Feria quinta}{}{Infra hebdomadam I Adventus}{3}
\cantus{A1F5VAM}{Ad Magn.}

\festum{}{Feria sexta}{}{Infra hebdomadam I Adventus}{3}
\cantus{A1F6VAM}{Ad Magn.}
\lineamagna

%%%% DOM II. ADVENTUS
\festum{}{Dominica II Adventus}{I classis}{Dominica II Adventus}{2}
\titulumrubrica{Ad I Vesperas}
\cantus{A1F7VR}{\rrrub}
\cantus{A1F7VAM}{Ad Magn.}

\titulumrubrica{Ad II Vesperas}
\cantus{A2F1VAM}{Ad Magn.}

%%%% INFRA HEBDOMADAM II ADVENTUS
\festum{}{Feria secunda}{}{Infra hebdomadam II Adventus}{3}
\cantus{A2F2VAM}{Ad Magn.}

\festum{}{Feria tertia}{}{Infra hebdomadam II Adventus}{3}
\cantus{A2F3VAM}{Ad Magn.}

\festum{}{Feria quarta}{}{Infra hebdomadam II Adventus}{3}
\cantus{A2F4VAM}{Ad Magn.}

\festum{}{Feria quinta}{}{Infra hebdomadam II Adventus}{3}
\cantus{A2F5VAM}{Ad Magn.}

\festum{}{Feria sexta}{}{Infra hebdomadam II Adventus}{3}
\cantus{A2F6VAM}{Ad Magn.}
\lineamagna

%%%% DOM III. ADVENTUS
\festum{}{Dominica III Adventus}{I classis}{Dominica III Adventus}{2}
\titulumrubrica{Ad I Vesperas}
\cantus{A2F7VR}{\rrrub}
\cantus{A2F7VAM}{Ad Magn.}

\titulumrubrica{Ad II Vesperas}
\cantus{A3F1VAM}{Ad Magn.}
\lineamagna

%%%%ANTIPHONAE MAJORES (``O'')
\rubrica{Debent sequentes septem antiphonæ quæ incipiunt per} O, \rubrica{die 17 decembris inchoari, et sic consequenter dici ad Magnificat vel ad memoriam de Adventu usque ad vigiliam Nativitatis Domini exclusive, servato dumtaxat hoc ordine quo positæ sunt. Et tunc in Vesperis nec preces dicuntur nec prostrationes fiunt, etiam si Officium de feria fuerit.}

\festum{}{Die 17 decembris}{}{Antiphonæ Majores}{3}
\cantus{1712VAM}{Ad Magn.}

\festum{}{Die 18 decembris}{}{Antiphonæ Majores}{3}
\cantus{1812VAM}{Ad Magn.}

\festum{}{Die 19 decembris}{}{Antiphonæ Majores}{3}
\cantus{1912VAM}{Ad Magn.}

\festum{}{Die 20 decembris}{}{Antiphonæ Majores}{3}
\cantus{2012VAM}{Ad Magn.}

\festum{}{Die 21 decembris}{}{Antiphonæ Majores}{3}
\cantus{2112VAM}{Ad Magn.}

\festum{}{Die 22 decembris}{}{Antiphonæ Majores}{3}
\cantus{2212VAM}{Ad Magn.}

\festum{}{Die 23 decembris}{}{Antiphonæ Majores}{3}
\cantus{2312VAM}{Ad Magn.}
\lineamagna

%%%% INFRA HEBDOMADAM III ADVENTUS
\festum{}{Feria secunda}{}{Infra hebdomadam III Adventus}{3}
\cantus{A3F2VAM}{Ad Magn.}
\rubrica{Vel ant.} O.

\festum{}{Feria tertia}{}{Infra hebdomadam III Adventus}{3}
\cantus{A3F3VAM}{Ad Magn.}
\rubrica{Vel ant.} O.

\festum{}{Feria quarta Quatuor Temporum}{}{Infra hebdomadam III Adventus}{3}
\cantus{A3F4VAM}{Ad Magn.}
\rubrica{Vel ant.} O.

\festum{}{Feria quinta}{}{Infra hebdomadam III Adventus}{3}
\cantus{A3F5VAM}{Ad Magn.}
\rubrica{Vel ant.} O.

\festum{}{Feria sexta Quatuor Temporum}{}{Infra hebdomadam III Adventus}{3}
\cantus{A3F6VAM}{Ad Magn.}
\rubrica{Vel ant.} O.
\lineamagna

%%%% DOM IV. ADVENTUS
\festum{}{Dominica IV Adventus}{I classis}{Dominica IV Adventus}{2}
\titulumrubrica{Ad I Vesperas}
\cantus{A3F7VR}{\rrrub}
\rubrica{Ad Magnificat antiphona} O.


\titulumrubrica{Ad II Vesperas}
\rubrica{Ad Magnificat antiphona} O.
\end{document}