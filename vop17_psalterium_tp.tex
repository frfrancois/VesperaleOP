% !TEX TS-program = lualatex
% !TEX encoding = UTF-8

\documentclass[VesperaleOP.tex]{subfiles}

\ifcsname preamble@file\endcsname
  \setcounter{page}{\getpagerefnumber{M-vop17_psalterium_tp}}
\fi

\begin{document}

\festum{}{Psalterium}{Tempori Paschale}{Psalterium T. P.}{0}
\lineaparva

%%%%DOMINICA
\festum{}{Dominica}{ad Vesperas}{Dominica T. P. ad Vesperas}{3}
\lineaparva

\cantus{F1PVA}{Ant. T. P.}
\psalmus[7tp]{109}
\psalmus{110}
\psalmus{111}
\psalmus{112}
\psalmus{113}

\titulumrubrica{Tempore Resurrectionis}
\cantus{PVH}{Hymnus}
\versiculus{Mane nobíscum, Dómine, allelúia.}{Quóniam advesperáscit, allelúia.}
\lineamagna

%%%%FERIA II
\festum{}{Feria II}{ad Vesperas}{Feria II ad Vesperas}{3}
\lineaparva

\cantus{F2PVA}{Ant. T. P.}
\psalmus{114}
\psalmus{115}
\psalmus{119}
\psalmus{120}
\psalmus{121}
\lineamagna

%%%%FERIA III
\festum{}{Feria III}{ad Vesperas}{Feria III ad Vesperas}{3}
\lineaparva

\cantus{F3PVA}{Ant. T. P.}
\psalmus{122}
\psalmus{123}
\psalmus{124}
\psalmus{125}
\psalmus{126}
\lineamagna

%%%%FERIA IV
\festum{}{Feria IV}{ad Vesperas}{Feria IV ad Vesperas}{3}
\lineaparva

\cantus{F4PVA}{Ant. T. P.}
\psalmus{127}
\psalmus{128}
\psalmus{129}
\psalmus{130}
\psalmus{131}
\lineamagna

%%%%FERIA V
\festum{}{Feria V}{ad Vesperas}{Feria V ad Vesperas}{3}
\lineaparva

\cantus{F5PVA}{Ant. T. P.}
\psalmus[1tp]{132}
\psalmus{135, i}
\psalmus{135, ii}
\psalmus{136}
\psalmus{137}
\lineamagna

%%%%FERIA VI
\festum{}{Feria VI}{ad Vesperas}{Feria VI ad Vesperas}{3}
\lineaparva

\cantus{F6PVA}{Ant. T. P.}
\psalmus[3tp]{138, i}
\psalmus{138, ii}
\psalmus{139}
\psalmus{140}
\psalmus{141}
\lineamagna

%%%%SABBATO
\festum{}{Sabbato}{ad Vesperas}{Sabbato ad Vesperas}{3}
\lineaparva

\cantus{SPVA}{Ant. T. P.}
\psalmus{143, i}
\psalmus{143, ii}
\psalmus{144, i}
\psalmus{144, ii}
\psalmus{144, iii}
\lineamagna

\end{document}