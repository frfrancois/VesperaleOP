% !TEX TS-program = lualatex
% !TEX encoding = UTF-8

\documentclass[VesperaleOP.tex]{subfiles}

\ifcsname preamble@file\endcsname
  \setcounter{page}{\getpagerefnumber{M-vop15_psalterium}}
\fi

\begin{document}

\festum{}{Psalterium}{per hebdomadam dispositum}{Psalterium}{0}

\festum{}{Dominica}{ad Vesperas}{Dominica ad Vesperas}{2}

\festum{}{Feria II}{ad Vesperas}{Feria II ad Vesperas}{2}

\rubrica{Psalmi ad Vesperas in hac Feria et in aliis Feriis assignati, cum appositis Antiphonis dicuntur in Feriali Officio, et in Festis quando sumendi sunt Psalmi feriales ad Vesperas.}

\cantus{F2A1}{1. Ant}
\psalmus{114}
\cantus{F2A2}{2. Ant}
\psalmus{115}
\cantus{F2A3}{3. Ant}
%\psalmus{119}
\cantus{F2A4}{4. Ant}
%\psalmus{120}
\cantus{F2A5}{5. Ant}
%\psalmus{121}

\titulumrubrica{Per Annum}

\cantus{F2H}{Hymnus}
\versiculus{Dirigátur, Dómine, orátio mea.}{Sicut incénsum in conspéctu tuo.}

\rubrica{Hymnus et \vvrub hic positi dicuntur tantum in Feriali Officio per Annum. Tempore autem Adventus, Quadragesimæ, Passionis et Paschali habentur proprii, qui pag. \pageref{A1H} notantur.}

\titulumrubrica{Per Annum}
\cantus{F2AM}{Ad Magn.}
%\canticum{Magnificat}

\rubrica{Antiphona ad Magnificat singulis Feriis apposita dicitur tantum in Feriali Officio per Annum, sed post Dominicam Septuagesimæ habentur propriæ ut ibidem notatur.}

\rubrica{In Festis Ant. ad Magnificat sumitur de Proprio vel Communi.}

\rubrica{Si tempus requirat, fit Memoria de Cruce, pag. \pageref{TODO}.}


\festum{}{Feria III}{ad Vesperas}{Feria III ad Vesperas}{2}

\festum{}{Feria IV}{ad Vesperas}{Feria IV ad Vesperas}{2}

\festum{}{Feria V}{ad Vesperas}{Feria V ad Vesperas}{2}

\festum{}{Feria VI}{ad Vesperas}{Feria VI ad Vesperas}{2}

\festum{}{Sabbato}{ad Vesperas}{Sabbato ad Vesperas}{2}

\end{document}