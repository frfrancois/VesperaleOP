% !TEX TS-program = lualatex
% !TEX encoding = UTF-8

\documentclass[VesperaleOP.tex]{subfiles}

\ifcsname preamble@file\endcsname
  \setcounter{page}{\getpagerefnumber{M-vop15_psalterium}}
\fi

\begin{document}

\festum{}{Psalterium}{per hebdomadam dispositum}{Psalterium}{0}

%%%%DOMINICA
\festum{}{Dominica}{ad Vesperas}{Dominica ad Vesperas}{2}
\lineaparva

\cantus{DVA1}{1. Ant}
\psalmus{109}
\cantus{DVA2}{2. Ant}
\psalmus{110}
\cantus{DVA3}{3. Ant}
\psalmus{111}
\cantus{DVA4}{4. Ant}
\psalmus{112}
\cantus{DVA5}{5. Ant}
\psalmus{113}

\titulumrubrica{Tempore Adventus}
\cantus{AVH}{Hymnus}
\versiculus{Roráte cæli désuper, et nubes pluant justum.}{Aperiátur terra, et gérminet Salvatórem.}

\titulumrubrica{Per Annum et tempore Septuagesimæ}
\cantus{DVH}{Hymnus}
\versiculus{Dirigátur, Dómine, orátio mea.}{Sicut incénsum in conspéctu tuo.}

\titulumrubrica{Tempore Quadragesimæ}
\cantus{Q1VH}{Hymnus}
\versiculus{Angelis suis Deus mandávit de te.}{Ut custódiant te in ómnibus viis tuis.}

\titulumrubrica{Tempore Passionis}
\cantus{Q5VH}{Hymnus}
\versiculus{Eripe me, Dómine, ab hómine malo.}{A viro iníquo éripe me.}


%%%FERIA II
\lineamagna
\festum{}{Feria II}{ad Vesperas}{Feria II ad Vesperas}{2}
\lineaparva

\rubrica{Psalmi ad Vesperas in hac Feria et in aliis Feriis assignati, cum appositis Antiphonis dicuntur in Feriali Officio, et in Festis quando sumendi sunt Psalmi feriales ad Vesperas.}

\cantus{F2VA1}{1. Ant}
\psalmus{114}
\cantus{F2VA2}{2. Ant}
\psalmus{115}
\cantus{F2VA3}{3. Ant}
\psalmus{119}
\cantus{F2VA4}{4. Ant}
\psalmus{120}
\cantus{F2VA5}{5. Ant}
\psalmus{121}

\titulumrubrica{Per Annum}

\cantus{F2VH}{Hymnus}
\versiculus{Dirigátur, Dómine, orátio mea.}{Sicut incénsum in conspéctu tuo.}

\rubrica{Hymnus et \vvrub hic positi dicuntur tantum in Feriali Officio per Annum. Tempore autem Adventus, Quadragesimæ, Passionis et Paschali habentur proprii, qui pag. \pageref{AVH} notantur.}

\titulumrubrica{Per Annum}
\cantus{F2VAM}{Ad Magn.}
%\canticum{Magnificat}{Beatæ Mariæ Virginis}{Lc. 1}

\rubrica{Antiphona ad Magnificat singulis Feriis apposita dicitur tantum in Feriali Officio per Annum, sed post Dominicam Septuagesimæ habentur propriæ ut ibidem notatur.}

\rubrica{In Festis Ant. ad Magnificat sumitur de Proprio vel Communi.}


%%%FERIA III
\lineamagna
\festum{}{Feria III}{ad Vesperas}{Feria III ad Vesperas}{2}
\lineaparva

\cantus{F3VA1}{1. Ant}
\psalmus{122}
\cantus{F3VA2}{2. Ant}
\psalmus{123}
\cantus{F3VA3}{3. Ant}
\psalmus{124}
\cantus{F3VA4}{4. Ant}
\psalmus{125}
\cantus{F3VA5}{5. Ant}
\psalmus{126}

\titulumrubrica{Per Annum}
\cantus{F3VAM}{Ad Magn.}


%%%FERIA IV
\lineamagna
\festum{}{Feria IV}{ad Vesperas}{Feria IV ad Vesperas}{2}
\lineaparva

\cantus{F4VA1}{1. Ant}
\psalmus{127}
\cantus{F4VA2}{2. Ant}
\psalmus{128}
\cantus{F4VA3}{3. Ant}
\psalmus{129}
\cantus{F4VA4}{4. Ant}
\psalmus{130}
\cantus{F4VA5}{5. Ant}
\psalmus{131}
\titulumrubrica{Per Annum}
\cantus{F4VAM}{Ad Magn.}


%%%FERIA V
\lineamagna
\festum{}{Feria V}{ad Vesperas}{Feria V ad Vesperas}{2}
\lineaparva

\cantus{F5VA1}{1. Ant}
\psalmus{132}
\cantus{F5VA2}{2. Ant}
\psalmus{135, i}
\cantus{F5VA3}{3. Ant}
\psalmus{135, ii}
\cantus{F5VA4}{4. Ant}
\psalmus{136}
\cantus{F5VA5}{5. Ant}
\psalmus{137}
\titulumrubrica{Per Annum}
\cantus{F5VAM}{Ad Magn.}

\lineamagna
\festum{}{Feria VI}{ad Vesperas}{Feria VI ad Vesperas}{2}
\lineaparva


%%%FERIA VI
\cantus{F6VA1}{1. Ant}
\psalmus{138, i}
\cantus{F6VA2}{2. Ant}
\psalmus{138, ii}
\cantus{F6VA3}{3. Ant}
\psalmus{139}
\cantus{F6VA4}{4. Ant}
\psalmus{140}
\cantus{F6VA5}{5. Ant}
\psalmus{141}
\titulumrubrica{Per Annum}
\cantus{F6VAM}{Ad Magn.}


%%%SABBATO
\lineamagna
\festum{}{Sabbato}{ad I Vesperas dominicæ}{Sabbato ad I Vesperas dominicæ}{2}
\lineaparva

\cantus{SVA1}{1. Ant}
\psalmus{143, i}
\cantus{SVA2}{2. Ant}
\psalmus{143, ii}
\cantus{SVA3}{3. Ant}
\psalmus{144, i}
\cantus{SVA4}{4. Ant}
\psalmus{144, ii}
\cantus{SVA5}{5. Ant}
\psalmus{144,  iii}

\titulumrubrica{Per Annum}

\cantus{SVH}{Hymnus}
\versiculus{Vespertína orátio ascéndat ad te, Dómine.}{Et descéndat super nos misericórdia tua.}

\rubrica{Ant. ad Magnificat ut in Proprio de Tempore, præterquam in Sabbatis ante Dom. II post Epiphaniæ usque ad Sabbatum ante Septuagesimam inclusive, in quibus dicitur sequens :}
\cantus{SVAM}{Ad Magn.}

\end{document}