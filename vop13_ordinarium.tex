% !TEX TS-program = lualatex
% !TEX encoding = UTF-8

\documentclass[VesperaleOP.tex]{subfiles}

\ifcsname preamble@file\endcsname
  \setcounter{page}{\getpagerefnumber{M-vop13_ordinarium}}
\fi

\begin{document}

\festum{}{Ordinarium divini Officii}{ad Vesperas}{Ordinarium}{0}
\lineaparva

%%%INCHOATIO VESPERARUM
\cantus{Deus, in adjutorium}{}

\rubrica{A Septuagesima usque ad Pascha, loco} Allelúia.\rubrica{, dicitur :}
\cantus{Laus tibi, Domine}{}
\lineaparva

\rubrica{Deinde, sub una vel quinque antiphonis, dicuntur quinque psalmi, prouti Officium occurens requirit.}

\rubrica{Repetita vero post ultimum psalmum antiphona, dicuntur capitulum, (responsorium), hymnus et versiculus, prouti Officium occurens exigit.}

\rubrica{In I Vesperis cujusvis festi post capitulum dicendum est responsorium ; pariter in I Vesperis dominicæ, quando historia dominicalis in sua prima dominica inchoatur.}

\rubrica{Post versiculum dicitur cum antiphona convenienti sequens}

\festum{}{Canticum beatæ Mariæ Virginis}{juxta octo modos cantus ecclesiastici et eorum differentias}{}{2}

%%%MAGNIFICAT
%\titulumrubrica{I a}
%\cantus{M_1a}{}
%\titulumrubrica{I b}
%\cantus{M_1b}{}
%\titulumrubrica{I c}
%\cantus{M_1c}{}
\lineaparva
\rubrica{Repetita post canticum antiphona, si preces non fuerint dicendæ, statim subjungitur} Dóminus vobíscum \rubrica{cum oratione et reliquis omnibus usque ad finem Horæ, ut infra.}
\lineamagna

%%%PRECES FERIALES
\titulumrubrica{Modum dicendi preces}

\cantus{Kyrie}{}

Pater noster \rubrica{secreto.}

\cantus{Et ne nos inducas}{}

\lineamagna

%%%ORATIO
\rubrica{Finitis precibus, vel, si preces dicendæ non fuerint, repetita post canticum antiphona, subjungitur:}

\cantus{Dominus vobiscum_1}{}

\rubrica{Et dicitur oratio conveniens.}

\rubrica{Finita oratione et responso} Amen.\rubrica{, si memoriæ fuerint faciendæ, fiunt per antiphonam, versiculum et orationem absque} Dóminus vobíscum\rubrica{, præmisso} Orémus.

\rubrica{Finita oratione vel finitis orationibus (si plures dicendæ occurrerint), sujungitur:}

\cantus{Dominus vobiscum_2}{}

\rubrica{Addito} Dóminus vobíscum.\rubrica{, ut supra, cantetur} Benedicámus Dómino.
\lineamagna

%%%BENEDICAMUS DOMINO
\festum{}{Modus cantandi Benedicamus ad Vesperas}{Extra Tempus Paschale}{}{2}

\rubrica{In festis I classis Domini et Dedicationis:}

\cantus{B1}{}

\rubrica{In festis I classis B. Mariæ V. et in die Octava Nativitatis:}

\cantus{B2}{}

\rubrica{In festis I classis Sanctorum:}

\cantus{B3}{}

\rubrica{In festis II classis, in Dominicis I classis, ac Dom. infra octavam Nativitatis:}

\cantus{B4}{}

\rubrica{In Dominicis II classis, in festis III classis ubi psalmi dominici dicuntur ad Laudes et in diebus infra octavam Nativitatis:}

\cantus{B5}{}

\rubrica{In festis III classis ubi psalmi feriales dicuntur ad Laudes, in Vesperis feriarum cum ant. ''0'', et in feriis a die 2 usque ad diem 12 januarii inclusive:}

\cantus{B6}{}

\rubrica{In feriis (extra tempus natalicio):}

\cantus{B7}{}
\lineaparva

\festum{}{}{Tempore Paschali}{}{2}

\rubrica{In festis et in Dominicis, ac infra Octavas:}

\cantus{B8}{}

\rubrica{In feriis:}

\cantus{B9}{}

\lineamagna

\rubrica{Responso} Deo grátias \rubrica{si non sequitur aliqua Hora, sine nota, sonore, et humili voce dicatur:} Fidélium ánimæ per misericórdiam Dei requiéscant in pace.  \rubrica{Et eodem modo respondeatur:} Amen.

\end{document}